\chapter{Extensiones de cuerpos}

\section{Extensiones de cuerpos}

\begin{definition}{Extensión de cuerpos}{def_extension}
Sea \(K\) un cuerpo. Una extensión de \(K\) es un cuerpo \(L\) que contiene a \(K\) como subcuerpo. En tal caso decimos que \(L / K\) es una extensión de cuerpos o simplemente una extensión.
\end{definition}

Observe que si \(L / K\) es una extensión de cuerpos, entonces \(L\) tiene una estructura natural de espacio vectorial sobre \(K\). Los vectores son los elementos de \(L\) y los escalares son los elementos de \(K\), la suma de vectores es la suma en \(L\) y el producto de escalares por vectores está bien definido puesto que los elementos de \(K\) está en \(L\). Denotaremos este espacio vectorial como \(L_{K}\) y una base de la extensión \(L / K\) es simplemente una base de este espacio vectorial.

\begin{definition}{Grado de una extensión}{}
La dimensión de \(L_{K}\) se llama grado de la extensión \(L / K\) y se representa por \([L:K]\). O sea
\[[L:K] = \dim_K(L).\]
\end{definition}

\begin{example}{Extensión de los reales}{extension_reales}
Tomemos $K = \R$, $L = \C$. Entonces $L/K$ es una extensión, en este caso los vectores del espacio vectorial son números complejos, y para construir una combinación lineal de ellos solo podemos emplear escalares reales.

El conjunto $B = \{1,i\}$ genera a $L_K$: cualquier $z \in L_K$ se puede expresar como
\[
z = \operatorname{Re}(z) 1 + \operatorname{Im}(z) i, \quad \operatorname{Re}(z), \operatorname{Im}(z) \in \R.
\]
Además, si $a,b\in \R$ cumplen $a 1 + b i = 0 \implies a,b = 0$, por lo que $B$ es una base. De aquí deducimos que $[\C:\R] = 2$.
\end{example}

Decimos que \(L / K\) es una extensión finita si \([L:K] < \infty\). Obsérvese que si \(L / K\) es una extensión de grado \(n\) entonces la base del espacio vectorial $L_K$ tiene $n$ vectores, por tanto, según un resultado conocido de algebra lineal,
\[
L_{K} \simeq K^{n}.
\]
De aquí deducimos que, \(|L| = |K|^{n}\). Gracias a este resultado obtenemos la siguiente propisición.

\begin{proposition}{}{extensiones_finitas}
Sea $L/K$ una extensión finita.
\begin{enumerate}
    \item Si \(K\) es finito de orden \(q\), entonces \(L\) es finito de orden \(q^{n}\).
    \item Si \(K\) es infinito entonces \(L\) tiene el mismo cardinal que \(K\).
\end{enumerate}
\end{proposition}

\subsection{Ejemplos de extensiones de cuerpos}

\begin{example}{}{extension_1}
Si \(L / K\) es una extensión de cuerpos, entonces \([L:K] = 1\) si y solo si \(K = L\).
\end{example}

\begin{proofbox}
Es inmediato que si $K=L$ entonces una base de $L_K$ es $B=\{1\}$, por lo que $[L:K]=1$. 

Por otro lado, si las bases de $L_K$ tiene un solo elemento, podemos fijar una base $B=\{\alpha\}$, $\alpha \neq 0$. En concreto, la identidad debe expresarse como combinación lineal de elementos de esa base, es decir,
\[
1 = \lambda \alpha
\]
para cierto $\lambda \in K$, pero entonces debe ser $\alpha = \lambda^{-1} \in K$, por lo que cualquier elemento $a \in L$ es combinación de un escalar $b \in K$ con $\alpha$
\[
a = b \alpha \in K \quad \text{ ya que } \alpha \in K
\]
por tanto, $L\subseteq K \implies L = K$.
\end{proofbox}

\begin{example}{}{extension_2}
Como hemos visto en el Ejemplo \ref{ex:extension_reales}, \(\mathbb{C} / \mathbb{R}\) es una extensión finita de grado 2.
\end{example}

\begin{example}{}{extension_3}
\(\mathbb{R} / \mathbb{Q}\) y \(\mathbb{C} / \mathbb{Q}\) son extensiones de grado infinito.
\end{example}

\begin{proofbox}
Para verlo, supongamos que fueran de grado finito. Entonces, como \(\mathbb{Q}\) es infinito, por el apartado 2 de la Proposición \ref{prop:extensiones_finitas}, \(\mathbb{R}\) y \(\mathbb{C}\) deberían tener el mismo cardinal que $\mathbb{Q}$. Sin embargo, sabemos que $\R,\C$ tienen mayor cardinal que \(\mathbb{Q}\), luego ambas extensiones deben ser de grado infinito.
\end{proofbox}

\begin{example}{}{extension_4}
Si \(n \in \mathbb{Q}\), entonces \(\mathbb{Q}(\sqrt{n}) = \{a + b\sqrt{n}:a,b\in \mathbb{Q}\}\) es una extensión que tiene grado 1 si \(n\) es un cuadrado de un número racional y grado 2 en caso contrario pues, en el segundo caso, \(\{1,\sqrt{n}\}\) es una base de \(\mathbb{Q}(\sqrt{n}) / \mathbb{Q}\).
\end{example}

\begin{example}{}{extension_5}
El cuerpo de fracciones \(K(X)\) del anillo de polinomios \(K[X]\) es una extensión de \(K\) de grado infinito.
\end{example}

\begin{proofbox}
Por un resultado sobre anillos, como $K$ es un cuerpo, en concreto es un dominio, y entonces $K[X]$ también lo es. Por tanto, tiene sentido considerar el cuerpo de fracciones de $K[X]$, que denotamos $K(X)$. Claramente, $K \subseteq K(X)$\footnote{También es cierto que $K \subseteq K[X]$, pero $K[X]$ no tiene por qué ser un cuerpo.}. Para ver que la extensión es de grado infinito encontraremos un conjunto infinito de elementos linealmente independientes. De esto se deduce que cualquier base de $K(X)$ debe tener infinitos elementos. Sea 
\[
C = \{1, X^{-1}, X^{-2}, \dots\},
\]
consideremos una combinación lineal cualquiera de $m$ elementos de $C$:
\[
P = a_1 X^{-n_1} + a_2 X^{-n_2} + \dots + a_m X^{-n_m}, \quad n_1 \leq \dots \leq n_m
\]
entonces,
\[
P = 0 \iff X^{n_m} P = 0 \iff a_1 X^{n_m-n_1} + \dots + a_m = 0 \iff \forall i,\ a_i = 0
\]
ya que $X^{n_m} P$ es un polinomio en $K$ y solo puede ser cero si todos sus coeficientes son $0$.
\end{proofbox}

\subsection{Torres de cuerpos y propiedades}

\begin{definition}{Torre de extensiones de cuerpos}{}
Una torre de extensiones de cuerpos es una sucesión
\[
K_{1}\subseteq K_{2}\subseteq \dots \subseteq K_{n}
\]
de cuerpos, cada uno subcuerpo de los posteriores. Cada extensión \(K_{i + 1} / K_{i}\) se llama subextensión de la torre.
\end{definition}

\begin{definition}{Clase de extensiones multiplicativa}{}
Una clase de extensiones $\mathcal{C} = \{L_i / K_i\}_{i \in I}$ se dice multiplicativa si para cada torre $K_1 \subseteq K_2 \subseteq K_3$ se cumple
\[
K_3 / K_1 \in \mathcal{C} \iff K_3 / K_2 \in \mathcal{C} \text{ y } K_2 / K_1 \in \mathcal{C}.
\]
\end{definition}

Más adelante veremos ejemplos interesantes de torres de cuerpos y clases de extensiones multiplicativas.

\begin{definition}{$K$-homomorfismo}{}
Si \(L_{1}\) y \(L_{2}\) son dos extensiones de \(K\), entonces un homomorfismo de \(L_{1} / K\) en \(L_{2} / K\) (también llamado \(K\)-homomorfismo) es un homomorfismo de cuerpos \(f:L_{1}\to L_{2}\) tal que para todo \(a\in K\), \(f(a) = a\).

Un endomorfismo de una extensión \(L / K\) es un homomorfismo de \(L / K\) en si misma. Un isomorfismo de extensiones (o \(K\)-isomorfismo) es un homomorfismo de extensiones que es isomorfismo de cuerpos y un automorfismo de extensiones (o \(K\)-automorfismo) es un isomorfismo de una extensión de \(K\) en si misma.
\end{definition}

Obsérvese que el conjunto de los automorfismos de una extensión \(L / K\) es un grupo que llamaremos grupo de Galois de \(L / K\), en el que el producto es la composición de aplicaciones, y que denotaremos por
\[
\Gal(L / K).
\]

\begin{definition}{Subextensión}{}
Una subextensión de una extensión de cuerpos \(L / K\) es un subcuerpo \(M\) de \(L\) que contiene a \(K\):
\[
K \subseteq M \subseteq L.
\]

Dos extensiones \(L_{1}\) y \(L_{2}\) de un cuerpo \(K\) se dice que son admisibles si existe un cuerpo \(L\) que es extensión de \(L_{1}\) y \(L_{2}\), o lo que es lo mismo, si ambas son subextensiones de una extensión común \(L / K\).

\begin{center}
\begin{tikzcd}[column sep=small, row sep=large]
& L & \\
L_1 \arrow[ur, hook] & & L_2 \arrow[ul, hook'] \\
& K \arrow[ul, hook] \arrow[ur, hook'] &
\end{tikzcd}
\end{center}
\end{definition}

Por convenio, en todos los cuerpos suponemos que \(0\neq 1\). Eso implica que todos los homomorfismos entre cuerpos son inyectivos.
\begin{proofbox}
Sea $f : K \to L$ un homomorfismo de cuerpos. Sea $x \in \ker{f}$, si suponemos que $x \neq 0$, entonces
\[
1_L = f(1_K) = f(x x^{-1}) = f(x)f(x^{-1}) = 0 
\]
lo cual es contradictorio. Por tanto, $\ker{f} = \{1_K\}$, por lo que
\[
f(x) = f(y) \iff 0 = f(y)-f(x) = f(y-x) \iff y-x = 0 \iff y = x.
\]
\end{proofbox}

Además los \(K\)-homomorfismos son homomorfismos de \(K\)-espacios vectoriales. De esta forma siempre que exista un homomorfismo de cuerpos \(f:K\to L\), el cuerpo \(L\) contiene un subcuerpo isomorfo a \(K\), la imagen \(f(K)\) de \(f\).

Por otro lado \(K\) admite una extensión isomorfa a \(L\), a saber el conjunto \(K\cup (L\setminus f(K))\), en el que se define el producto de la forma obvia. Abusaremos a menudo de la notación y cada vez que tengamos un homomorfismo de cuerpos \(f:K\to L\), simplemente consideraremos \(K\) como subcuerpo de \(L\), identificando los elementos de \(K\) y \(f(K)\), a través de \(f\).

Veamos ahora diversas propiedades de los $K$-homomorfismos.

\begin{proposition}{Homomorfismos y grados}{}
Sean \(L_{1}\) y \(L_{2}\) extensiones de \(K\). Si existe un \(K\)-homomorfismo de cuerpos \(\varphi: L_{1} \to L_{2}\), entonces \([L_{1}:K] \leq [L_{2}:K]\).
\end{proposition}

\begin{proofbox}
Todo homomorfismo de cuerpos es inyectivo. Como \(\varphi\) es \(K\)-lineal, es una transformación lineal inyectiva de \(L_{1}\) a \(L_{2}\), considerados como \(K\)-espacios vectoriales. Por tanto,
\[
\dim_K L_{1} \leq \dim_K L_{2},
\]
es decir, \([L_{1}:K] \leq [L_{2}:K]\).
\end{proofbox}

\begin{proposition}{Endomorfismos de extensiones finitas}{}
Todo endomorfismo \(K\)-lineal \(\sigma: L \to L\) de una extensión finita \(L/K\) es un automorfismo.
\end{proposition}

\begin{proofbox}
\(\sigma\) es un homomorfismo de cuerpos, luego inyectivo. Como \(L/K\) es de dimensión finita, toda transformación lineal inyectiva \(L \to L\) es también sobreyectiva. Por tanto, \(\sigma\) es biyectivo, es decir, un automorfismo.
\end{proofbox}

\begin{proposition}{Transitividad de grados}{}
Sea \(K \subseteq E \subseteq L\) una torre de cuerpos y sean \(B\) una base de \(E\) sobre \(K\) y \(B'\) una base de \(L\) sobre \(E\). Entonces:
\begin{enumerate}
    \item[(a)] \(A = \{bb' : b \in B, b' \in B'\}\) es una base de \(L\) sobre \(K\).
    \item[(b)] En particular, \([L:K] = [L:E][E:K]\).
    \item[(c)] La clase de extensiones finitas es multiplicativa.
\end{enumerate}
\end{proposition}

\begin{proofbox}
\begin{enumerate}
    \item[(a)] Veamos primero que es conjunto generador. Dado \(l \in L\), se escribe \(l = \sum_{i} e_i b_i'\) con \(e_i \in E\), \(b_i' \in B'\). Cada \(e_i = \sum_{j} k_{ij} b_{ij}\) con \(k_{ij} \in K\), \(b_{ij} \in B\). Luego 
    \[
    l = \sum_{i,j} k_{ij} b_{ij} b_i',
    \]
    combinación de elementos de \(A\).
    
    Para la independencia lineal, supongamos \(\sum_{b\in B, b'\in B'} k_{b,b'} bb' = 0\) con \(k_{b,b'} \in K\). Fijado \(b'\), sea \(e_{b'} = \sum_{b} k_{b,b'} b \in E\). Entonces \(\sum_{b'} e_{b'} b' = 0\). Como \(B'\) es linealmente independiente sobre \(E\), \(e_{b'} = 0\) para todo \(b'\). Como \(B\) es linealmente independiente sobre \(K\), \(k_{b,b'} = 0\) para todo \(b,b'\).
    
    \item[(b)] Se tiene \(|A| = |B| \cdot |B'|\), luego de (a) deducimos
    \[
    [L:K] = |A| = |B| \cdot |B'| = [E:K] \cdot [L:E].
    \]
    
    \item [(c)]La clase \(\mathcal{C} = \{L/K \mid [L:K] < \infty\}\) es multiplicativa: en una torre \(K \subseteq E \subseteq L\),
    \[
    L/K \in \mathcal{C} \iff E/K \in \mathcal{C} \text{ y } L/E \in \mathcal{C}
    \]
    esto se sigue inmediatamente de (b).
\end{enumerate}
\end{proofbox}

\begin{proposition}{Compuesto de dos extensiones admisibles}{compuesto}
Si \(L_{1}\) y \(L_{2}\) son extensiones admisibles de \(K\) y \(L\) es un cuerpo que contiene a \(L_{1}\) y \(L_{2}\) como subcuerpos, entonces
\[
L_{1}L_{2} = \left\{ \frac{a_{1}b_{1} + \cdots + a_{n}b_{n}}{a_{1}'b_{1}' + \cdots + a_{n}'b_{n}'} : a_i, a_i' \in L_{1}, b_i, b_i' \in L_{2}, \sum a_i' b_i' \neq 0 \right\}
\]
es el menor subcuerpo de \(L\) que contiene a \(L_{1}\) y \(L_{2}\), lo llamamos compuesto de $L_1$ y $L_2$ en $L$.
\end{proposition}

\begin{proofbox}
Denotemos por \(F\) al conjunto de la derecha, que claramente está contenido en $L$. Veamos primero que es un cuerpo.
\begin{itemize}
    \item Claramente contiene a $0 = \frac{0_{L_1}0_{L_2}}{1_{L_1}1_{L_2}}$ y $1 = \frac{1_{L_1}1_{L_2}}{1_{L_1}1_{L_2}}$.
    \item Dados \(x, y \in F\), sean 
    \[
    x = \frac{\sum_{i=1}^n a_i b_i}{\sum_{i=1}^n a_i' b_i'}, \quad
    y = \frac{\sum_{j=1}^m c_j d_j}{\sum_{j=1}^m c_j' d_j'}
    \]
    con \(a_i, a_i', c_j, c_j' \in L_1\), \(b_i, b_i', d_j, d_j' \in L_2\).
    \item Suma:
    \[
    x + y = \frac{(\sum a_i b_i)(\sum c_j' d_j') + (\sum c_j d_j)(\sum a_i' b_i')}{(\sum a_i' b_i')(\sum c_j' d_j')}
    \]
    que está en \(F\) porque numerador y denominador son sumas de productos \(ab\) con \(a \in L_1, b \in L_2\).
    \item Producto: 
    \[
    xy = \frac{(\sum a_i b_i)(\sum c_j d_j)}{(\sum a_i' b_i')(\sum c_j' d_j')}
    \]
    también de la misma forma.
    \item Inverso multiplicativo: si \(x \neq 0\), entonces \(x^{-1} = \frac{\sum a_i' b_i'}{\sum a_i b_i} \in F\).
\end{itemize}

Veamos ahora que \(F\) contiene a \(L_1\) y \(L_2\):
\begin{itemize}
    \item Para \(a \in L_1\), $a = \frac{a 1_{L_2}}{1_{L_1}1_{L_2}}$.
    \item Para \(b \in L_2\), $b = \frac{1_{L_1} b}{1_{L_1}1_{L_2}}$.
\end{itemize}

Finalmente, veamos que \(F\) es el menor. Sea \(F'\) un subcuerpo de \(L\) que contiene \(L_1\) y \(L_2\). Entonces \(F'\) contiene todas las sumas finitas \(\sum a_i b_i\) con \(a_i \in L_1, b_i \in L_2\), y también sus cocientes. Luego \(F \subseteq F'\).
\end{proofbox}

\begin{proposition}{Subanillo y subcuerpo generados}{cuerpo_generado}
Sean \(L/K\) una extensión de cuerpos y \(S \subseteq L\). Entonces:
\begin{enumerate}
    \item[(a)] El menor subanillo de \(L\) que contiene a \(K\) y a \(S\) es
    \[
    K[S] = \{ p(s_1,\dots,s_n) \mid n \in \mathbb{N}, p \in K[X_1,\dots,X_n], s_i \in S \}.
    \]
    \item[(b)] El menor subcuerpo de \(L\) que contiene a \(K\) y a \(S\) es
    \[
    K(S) = \left\{ \frac{p(s_1,\dots,s_n)}{q(s_1,\dots,s_n)} \mid n \in \mathbb{N}, p,q \in K[X_1,\dots,X_n], s_i \in S, q(s_1,\dots,s_n) \neq 0 \right\}.
    \]
\end{enumerate}
\end{proposition}

\begin{proofbox}
\begin{enumerate}
    \item[(a)] Denotemos \(R = \{ p(s_1,\dots,s_n) \mid \dots \}\). 
    \begin{itemize}
        \item Claramente \(K \subseteq R\) (polinomios constantes) y \(S \subseteq R\) (polinomios \(X_i\)). 
        \item \(R\) es cerrado bajo suma y producto: dados dos elementos $x,y$
        \[
        x = p(s_1, \dots, s_n), y = q(s'_1, \dots s'_m)
        \]
        juntamos los \(s_i, s'_j\) usados y podemos ver $x,y$ como polinomios en $n+m$ variables, donde las variables son $\{s_1, \dots, s_n, s'_1, \dots, s'_m\}$. Como la suma y el producto de polinomios en esas $n+m$ variables da polinomios del mismo tipo, $R$ es cerrado bajo suma y producto.
        \item $R$ contiene al $1$ de $L$ puesto que este es el mismo $1$ de $K$, que es un polinomio constante.
    \end{itemize}
    Luego \(R\) es un subanillo que contiene \(K\) y \(S\). Si \(R'\) es otro subanillo con \(K \cup S \subseteq R'\), entonces \(R'\) contiene todos las sumas y productos de elementos de $S$ y $K$, es decir, todos los polinomios en elementos de \(S\), luego \(R \subseteq R'\), por tanto, \(R = K[S]\).

    \item[(b)] Sea \(F = \{ p(s_1,\dots,s_n)/q(s_1,\dots,s_n) \mid \dots \}\). 
    \begin{itemize}
        \item \(F\) es un subcuerpo: suma, producto e inversos se reducen a operaciones con polinomios en las variables $s_i \in S$, igual que en el apartado anterior. Hacemos solo el caso de los inversos, dado $x \in F \setminus \{0\}$
        \[
        x = \frac{p(s_1,\dots,s_n)}{q(s_1,\dots,s_n)},\quad p(s_1,\dots,s_n) \neq 0
        \]
        de donde vemos que $ x^{-1} = \frac{q(s_1,\dots,s_n)}{p(s_1,\dots,s_n)} \in F$.
        \item \(K \cup S \subseteq F\) por el mismo razonamiento del caso anterior tomando como cociente el polinomio constantemente igual a la unidad. 
        \item Si \(F'\) es un subcuerpo con \(K \cup S \subseteq F'\), entonces \(F'\) contiene todos los polinomios \(p(s_1,\dots,s_n)\) y sus cocientes, luego \(F \subseteq F'\). 
    \end{itemize}
    Por tanto, \(F = K(S)\).
\end{enumerate}
\end{proofbox}

Analicemos el contenido de la Proposición \ref{prop:cuerpo_generado}. Si \(L / K\) es una extensión y \(S\) es un subconjunto de \(L\), entonces \(K[S]\) denota el menor subanillo de \(L\) que contiene a \(K\) y lo llamamos subanillo de \(L\) generado por \(K\) y \(S\). Por otro lado, el subcuerpo \(K(S)\) se llama extensión de \(K\) generada por \(S\). También diremos que \(K(S)\) es el cuerpo que se obtiene adjuntando a \(K\) los elementos de \(S\). Notemos que aunque $S$ no tenga ninguna estructura, siempre es posible tomar el producto de elementos de $S$ y elementos de $K$, así como inversos, puesto que todos los elementos con los que se trata se encuentran dentro del cuerpo $L$.

Observando que la intersección de subcuerpos de un cuerpo \(L\) es otro subcuerpo de \(L\), se tiene que \(K(S)\) es la intersección de todos los subcuerpos de \(L\) que contienen a \(K\) y a \(S\). Obsérvese que si \(S_{1}\) y \(S_{2}\) son dos subconjuntos de \(L\) entonces
\[ 
K(S_{1})K(S_{2}) = K(S_{1}\cup S_{2}).
\]
\begin{proofbox}
    En primer lugar, $K(S_1)K(S_2)$ es un cuerpo que contiene a $K,S_1,S_2$, por lo que $K(S_{1}\cup S_{2}) \subseteq K(S_{1})K(S_{2})$. 
    Para la otra inclusión basta notar que cualquier $x \in K(S_1 \cup S_2)$ es de la forma
    \[
    x = \frac{p(s_{11}, \dots, s_{1n}, s_{21}, \dots, s_{2m})}{q(s_{11}, \dots, s_{1n}, s_{21}, \dots, s_{2m})}
    \]
    con los $s_{1i} \in S_1, s_{2j} \in S_2$, pero podemos ver tanto $p$, $q$ como producto de dos polinomios, uno en las variables $s_{1i}$ y otro en las variables $s_{2j}$, luego
    \[
    x = \frac{p_1(s_{11}, \dots, s_{1n})}{q_1(s_{11}, \dots, s_{1n})}\frac{p_2(s_{21}, \dots, s_{2m})}{q_2(s_{21}, \dots, s_{2m})} \in K(S_1)K(S_2).
    \]
\end{proofbox}

De la misma forma, si \(L_{1} / K\) y \(L_{2} / K\) son dos subextensiones de \(L\), entonces \(L_{1}L_{2}\) es la intersección de todos los subcuerpos de \(L\) que contienen a \(L_{1}\cup L_{2}\) y por tanto
\[
L_{1}L_{2} = K(L_{1}\cup L_{2}).
\]

Por otro lado, el concepto de compuesto de dos subextensiones, presentado en la Proposición \ref{prop:compuesto}, se puede generalizar de forma obvia a una familia arbitraria de subextensiones. Si \(C\) es una familia de subextensiones de \(L / K\) entonces el compuesto de \(C\) es el menor subcuerpo de \(L\) que contiene a todos los elementos de \(C\) y coincide con la intersección de todos los subcuerpos de \(L\) que contienen todos los elementos de \(C\) y con \(K(\cup_{E\in C}E)\). Si \(C = \{L_{1} / K,\ldots ,L_{n} / K\}\), entonces el compuesto de \(C\) se denota por \(L_{1}\dots L_{n}\) y está formado por todos los elementos de la forma
\[
\frac{\sum_{i = 1}^{m}a_{1i}\dots a_{ni}}{\sum_{i = 1}^{m}b_{1i}\dots b_{ni}}
\]
con \(m\) arbitrario, \(a_{ji},b_{ji}\in L_{i}\) y \(\sum_{i = 1}^{m}b_{1i}\dots b_{ni}\neq 0\).

Un caso importante se presenta cuando el conjunto $S$ es finito. Si \(S = \{a_{1},\ldots ,a_{n}\}\), entonces escribimos \(K[S] = K[a_{1},\ldots ,a_{n}]\) y \(K(S) = K(a_{1},\ldots ,a_{n})\). La siguiente definición muestra la importancia del caso $S$ finito.

\begin{definition}{Extensión finitamente generada y extensión simple}{}
Decimos que \(L / K\) es una extensión finitamente generada si existen \(a_{1},\ldots ,a_{n}\in L\) tales que \(L = K(a_{1},\ldots ,a_{n})\) y que es simple si \(L = K(a)\) para algún \(a\in L\). En este último caso decimos que \(a\) es un elemento primitivo de \(L / K\).
\end{definition}

\begin{remark}
Por lo general, una extensión finitamente generada no tiene que ser finita. En el Ejemplo \ref{ex:extension_5} vimos que $K(X)$ es una extensión de $K$ de grado infinito, pero esta extensión es finitamente generada, de hecho, es simple.

Por otro lado, el lector podrá comprobar fácilmente que toda extensión finita es finitamente generada. Para ello, solo hay que probar que dada una base $B = \{b_1, \dots, b_n\}$ de $L_K$, entonces $K(b_1, \dots, b_n) = L$.
\end{remark}

Recordemos ahora el Ejemplo \ref{ex:extension_4}, en el que vimos que si \(n\in \mathbb{Q}\) no es un cuadrado de un número racional, entonces \(\mathbb{Q}(\sqrt{n})\) es una extensión de \(\mathbb{Q}\) de grado 2. De hecho, notemos que en este caso $\Q[\sqrt{n}] = \Q(\sqrt{n})$, ya que todo elemento de $\Q(\sqrt{n})$ se puede expresar de la forma $a + b \sqrt{n}$ con $a,b \in \Q$.

\begin{proofbox}
Si consideramos un elemento arbitrario de $\Q(\sqrt{n})$, este es un cociente de polinomios en $\sqrt{n}$, es decir, un elemento de la forma
\[
\frac{p(\sqrt{n})}{q(\sqrt{n})},
\]
con \(p(X), q(X) \in \Q[X]\) y \(q(\sqrt{n}) \neq 0\). Como \(\sqrt{n}^2 = n\), en realidad podemos suponer que \(p(X)\) y \(q(X)\) son polinomios de grado menor o igual que 1, pues todos los términos $\sqrt{n}^k$ con $k \geq 2$ se pueden reducir a términos de grado 0 o 1. Por tanto, todo elemento de \(\Q(\sqrt{n})\) se puede escribir como
\[
\frac{a + b \sqrt{n}}{c + d \sqrt{n}},
\] con \(a,b,c,d \in \Q\) y \(c + d \sqrt{n} \neq 0\). Si \(d = 0\), entonces \(\frac{a + b \sqrt{n}}{c} = \frac{a}{c} + \frac{b}{c}\sqrt{n}\). Si \(d \neq 0\), multiplicando numerador y denominador por \(c - d \sqrt{n}\) obtenemos
\[
\frac{a + b \sqrt{n}}{c + d \sqrt{n}} = \frac{(a + b \sqrt{n})(c - d \sqrt{n})}{c^2 - d^2 n} = \frac{ac - bd n}{c^2 - d^2 n} + \frac{bc - ad}{c^2 - d^2 n} \sqrt{n},
\]
que también es de la forma \(a' + b' \sqrt{n}\) con \(a',b' \in \Q\). Por tanto, \(\Q(\sqrt{n}) \subseteq \{a + b \sqrt{n} : a,b \in \Q\}\). La inclusión contraria es inmediata.
\end{proofbox}

Notemos que el factor esencial para que funcione la demostración anterior es que \(\sqrt{n}^2 = n \), es decir, $\sqrt{n}$ es una raíz del polinomio irreducible \(X^{2} - n\in \mathbb{Q}[X]\). De hecho, este hecho es general y se recoge en el siguiente lema.

\begin{lemma}{Propiedades de las raíces de polinomios irreducibles}{lem_raices_irreducibles}
Sea \(L / K\) una extensión. Si \(\alpha \in L\) es una raíz de un polinomio irreducible \(p\) de grado \(n\) en \(K[X]\) entonces

\begin{enumerate}
    \item[(1)] \(K[\alpha] = K(\alpha)\)
    \item[(2)] Si \(q\in K[X]\), entonces \(q(\alpha) = 0\) si y solo si \(p\) divide a \(q\) en \(K[X]\)
    \item[(3)] \(1,\alpha ,\alpha^{2},\ldots ,\alpha^{n - 1}\) es una base de \(K(\alpha)_{K}\). En particular, \([K(\alpha):K] = n\)
\end{enumerate}
\end{lemma}

\begin{proofbox}
    \begin{enumerate}
    \item[(1)] Consideremos el homomorfismo de evaluación en \(\alpha\) 
    \[
    S: K[X]\to L,\quad S(q) = q(\alpha)
    \]
    y sea \(I = \ker S = \{q\in K[X]:q(\alpha) = 0\}\).
    
    Notemos que \(I\) es un ideal propio de \(K[X]\): \(I \neq (0)\) puesto que $p(\alpha)=0$ y $p$ es irreducible, por tanto distinto de $0$, por otro lado \(I \neq K[X]\) pues \(1 \notin I\) (ya que \(1(\alpha) = 1 \neq 0\)). 
    
    Como \(\alpha\) es raíz de \(p\) se tiene \((p)\subseteq I\subset K[X]\). Pero \((p)\) es un ideal maximal de \(K[X]\), pues \(K[X]\) es un DIP y $p$ es irreducible. Concluimos que \(I = (p)\) y, del Primer Teorema de Isomorfía deducimos que \(K[\alpha] = \operatorname{Im}S\simeq K[X] / (p)\), que es un cuerpo pues \((p)\) es un ideal maximal de \(K[X]\). 
    
    Finalmente, recordemos que $K[\alpha]$ es el menor subanillo de $L$ que contiene a $K$ y $\{\alpha\}$, y además hemos visto que es un cuerpo. Por otro lado, cualquier cuerpo que contenga a $K$ y $\{\alpha\}$ es, en concreto, un subanillo que contiene a $K$ y $\{\alpha\}$, y por tanto es mayor que $K[\alpha]$, por lo que este debe ser el menor cuerpo que contiene a $K$ y $\{\alpha\}$. Esto prueba que $K[\alpha] = K(\alpha)$.

    \item[(2)] Si $q(\alpha) = 0$, entonces \(q\in \ker S = (p)\), luego \(p\) divide a \(q\). La implicación contraria es inmediata.

    \item[(3)] Si \(\beta \in K(\alpha)\), como $K(\alpha)=K[\alpha]$, entonces \(b = f(\alpha)\) para algún \(f\in K[X]\). Como el grado define una función euclídea en \(K[X]\), existen \(q,r\in K[X]\) tales que \(f = qp + r\) y \(m = \mathrm{gr}(r) < \mathrm{gr}(p) = n\). Entonces \(\beta = f(\alpha) = r(\alpha) = r_{0} + r_{1}\alpha +r_{2}\alpha^{2}\dots +r_{m}\alpha^{m}\). Esto prueba que \(1,\alpha ,\ldots ,\alpha^{n - 1}\) genera \(K(\alpha)_K\). Para demostrar que son linealmente independientes ponemos \(\sum_{i = 0}^{n - 1}a_{i}\alpha^{i} = 0\), con \(a_{i}\in K\). Entonces \(\alpha\) es raíz del polinomio \(a = \sum_{i = 0}^{n - 1}a_{i}X^{i}\), es decir \(a\in \ker S = (p)\). Como \(n = \mathrm{gr}(p) > \mathrm{gr}(a)\), deducimos que \(a = 0\), es decir \(a_{i} = 0\) para todo \(i\).
\end{enumerate}
\end{proofbox}

% \section{Adjunción de raíces}

% El siguiente teorema muestra que todos los polinomios no constantes tienen alguna raíz en algún cuerpo.

% \begin{theorem}{Teorema de Kronecker}{thm_kronecker}
% Si \(K\) es un cuerpo y \(p\in K[X]\setminus K\), entonces existe una extensión \(L\) de \(K\) que contiene una raíz de \(p\).
% \end{theorem}

% \begin{proofbox}
% Como \(p\) es un elemento no nulo ni invertible de \(K[X]\) y este es un DFU, \(p\) es divisible en \(K[X]\) por un polinomio irreducible y todas las raíces de este divisor son raíces de \(p\). Por tanto podemos suponer que \(p\) es irreducible. Eso implica que \((p)\) es un ideal maximal de \(K[X]\), pues este último es un DIP. Entonces \(L = K[X] / (p)\) es un cuerpo. La composición de la inclusión \(K\to K[X]\) y la proyección \(K[X]\to L = K[X] / (p)\) es un homomorfismo (inyectivo) de cuerpos y por tanto podemos considerar \(L\) como una extensión de \(K\). Para acabar la demostración basta ver que \(a = X + (p)\) es una raíz de \(p\). En efecto, \(p(a) = p(X + (p)) = p + (p) = (p)\), que es el cero del anillo \(L\).
% \end{proofbox}

% Por tanto, si \(p\in K[X]\) es un polinomio no constante, entonces existe una extensión \(L / K\) que contiene una raíz \(\alpha\) de \(p\) y \(K(\alpha)\) es la menor subextensión de \(L / K\) que contiene a \(\alpha\).

% Decimos que un polinomio \(p\in K[X]\setminus K\) es completamente factorizable sobre \(K\) si es producto de polinomios de grado 1, o lo que es lo mismo si \(p = a(X - \alpha_{1})\dots (X - \alpha_{n})\) para ciertos \(a,\alpha_{1},\ldots ,\alpha_{n}\in K\). En tal caso las raíces de \(p\) son \(\alpha_{1},\ldots ,\alpha_{n}\). Por ejemplo

% \[ X^{3} - 1 = (X - 1)(X^{2} + X + 1) = (X - 1)\left(X - \frac{-1 + \sqrt{-3}}{2}\right)\left(X - \frac{-1 - \sqrt{-3}}{2}\right) \]

% es completamente factorizable sobre \(\mathbb{C}\), pero no sobre \(\mathbb{Q}\) ni \(\mathbb{R}\). El Teorema de Kronecker afirma que cada polinomio no constante tiene una raíz en alguna extensión. De hecho podemos decir algo más.

% \begin{corollary}{Factorización completa en alguna extensión}{cor_factorizacion}
% Si \(K\) es un cuerpo y \(p\in K[X]\setminus K\), entonces \(p\) es completamente factorizable en alguna extensión de \(K\).
% \end{corollary}

% \begin{proofbox}
% Por inducción sobre el grado de \(p\). Si el grado de \(p\) es 1, no hay nada que demostrar. Si el grado de \(p\) es mayor que 1 entonces \(p\) tiene una raíz \(\alpha\) en alguna extensión \(E\) de \(K\). Entonces \(p = (X - \alpha)q\) para algún \(q\in E[X]\setminus E\). Por hipótesis de inducción \(q\) es completamente factorizable en alguna extensión \(L\) de \(E\), es decir \(q\) es producto de polinomios de \(L[X]\) de grado menor o igual que 1. Por tanto, también \(p\) es producto de polinomios de \(L[X]\) de grado menor o igual que 2.
% \end{proofbox}

% Nuestro objetivo principal es establecer un criterio de cuándo un polinomio es resoluble por radicales que es precisamente cuando sus raíces se puedan expresar en sucesivas extensiones en las que en cada paso se adjunta una raíz \(n\)-ésima de elementos del cuerpo anterior. Para formalizar esto introducimos las siguientes definiciones:

% \begin{definition}{Torre radical y extensión radical}{def_torre_radical}
% Una torre radical es una torre de cuerpos

% \[ E_0\subseteq E_1\subseteq \dots \subseteq E_n \]

% tales que para cada \(i = 1,\ldots ,n\), existen \(n_i\geq 1\) y \(\alpha_{i}\in E_{i}\) tal que \(E_{i} = E_{i - 1}(\alpha_{i})\) y \(\alpha^{n_i}\in E_{i - 1}\).

% Una extensión de cuerpos \(L / K\) se dice que es radical si existe una torre radical

% \[ K = E_0\subseteq E_1\subseteq \dots \subseteq E_n = L. \]

% Una ecuación polinómica \(P(X) = 0\), con \(P\in K[X]\), se dice que es resoluble por radicales sobre \(K\) si existe una extensión radical \(L / K\) tal que \(P\) es completamente factorizable en \(L\). En tal caso también se dice que el polinomio \(P\) es resoluble por radicales sobre \(K\).
% \end{definition}

% O sea si suponemos que \(P\in K[X]\) entonces \(P\) es resoluble por radicales sobre \(K\) si \(K\) tiene una extensión radical que contiene todas las raíces de \(P\) y queremos descubrir cuándo pasa eso. Para llegar a ello tenemos que recorrer un largo camino que se completará en los dos últimos capítulos.

% Recordemos que si \(\sigma :K\to E\) es un homomorfismo de anillos, entonces \(\sigma\) tiene una única extensión a un homomorfismo entre los anillos de polinomios, que seguiremos denotando por \(\sigma :K[X]\to E[X]\) tal que \(\sigma (X) = X\). Este homomorfismo se comporta bien sobre las raíces.

% \begin{lemma}{Comportamiento de homomorfismos con raíces}{lem_hom_raices}
% Sean \(\sigma :E\to L\) un homomorfismo de cuerpos y \(p\in E[X]\). Si \(\alpha\) es una raíz de \(p\) en \(E\) entonces \(\sigma (\alpha)\) es una raíz de \(\sigma (p)\).

% Si \(E / K\) y \(L / K\) son extensiones de un cuerpo \(K\), \(p\in K[X]\) y \(\sigma\) es un \(K\)-homomorfismo entonces \(\sigma\) se restringe a una aplicación inyectiva del conjunto de las raíces de \(p\) en \(E\) al conjunto de las raíces de \(p\) en \(L\).

% En particular, si \(E = L\) (es decir, si \(\sigma \in \operatorname{Gal}(L / K)\)), entonces esta restricción de \(\sigma\) es una permutación del conjunto de las raíces de \(p\) en \(L\).
% \end{lemma}

% \begin{proofbox}
% Si \(p = p_0 + p_1X + \dots +p_nX^n\), entonces

% \[ \sigma (p)(\sigma (\alpha)) = (\sigma (p_0) + \sigma (p_1)X + \sigma (p_1)X^2 +\dots +\sigma (p_n)X^n)(\sigma (\alpha)) \]
% \[ = \sigma (p_0) + \sigma (p_1)\sigma (\alpha) + \sigma (p_1)\sigma (\alpha)^2 +\dots +\sigma (p_n)\sigma (\alpha)^n \]
% \[ = \sigma (p_0 + p_1\alpha +p_1\alpha^2 +\dots +p_n\alpha^n) \]
% \[ = \sigma (p(\alpha)) = \sigma (0) = 0. \]

% Esto prueba la primera afirmación. Las otras dos afirmaciones son consecuencias inmediatas de la primera.
% \end{proofbox}

% \begin{lemma}{Lema de Extensión}{lem_extension}
% Sea \(\sigma :K_1\to K_2\) un homomorfismo de cuerpos y sea \(p\in K_1[X]\) un polinomio irreducible. Sean \(L_1 / K_1\) y \(L_2 / K_2\) dos extensiones de cuerpos y sean \(\alpha_1\in L_1\) y \(\alpha_2\in L_2\) con \(\alpha_1\) una raíz de \(p\).

% Entonces existe un homomorfismo \(\tilde{\sigma}:K_1(\alpha_1)\to K_2(\alpha_2)\) tal que \(\tilde{\sigma}|_{K_1} = \sigma\) y \(\tilde{\sigma} (\alpha_1) = \alpha_2\) si y solo si \(\alpha_2\) es una raíz del polinomio \(\sigma (p)\). En tal caso sólo hay un homomorfismo \(\tilde{\sigma}\) que satisfaga la condición indicada y si además, \(\sigma\) es un isomorfismo, entonces también \(\tilde{\sigma}\) es un isomorfismo.
% \end{lemma}